%
%  Bilim İnsanının Amerika'da PhD'ye Başvuru Rehberi source file 
%
\documentclass[12pt,a4paper]{article}
% Packages
\usepackage[turkish]{babel}
%\usepackage[utf8x]{inputenc} 
\usepackage[T1]{fontenc}
%\usepackage{ucs} 
\usepackage{tabularx}
%\usepackage{cclicenses} % Creative commons
\usepackage{hyperref}
\usepackage[normalsections,normalindent,normalleading,normalmargins]{savetrees}
\usepackage{fullpage}
\usepackage{fontspec}
\usepackage{xunicode}
\usepackage{xltxtra}
\newfontfeature{Microtype}{protrusion=default;expansion=default;}
\setromanfont{Linux Libertine O}
%\setromanfont[Microtype]{Garamond Premier Pro}
%\setromanfont[Ligatures={Common, Historical, Rare}]{Cardo}
% end of Packages

% new command for table of contents

\setlength{\parindent}{0pt}
\setlength{\parskip}{2ex plus 0.5ex minus 0.2ex}


%\parindent 1cm  % paragraf girdisinin boyunu belirler
%\parskip 0.2cm    % paragraflar arasi boslugu belirler
\topmargin 0.15cm   % sayfa üstünde ne kadar bosluk birakilacagini belirler
\oddsidemargin 0cm  %sayfanin sol boslugu, tek numarali sayfa
\evensidemargin 1cm  %sayfanin sag boslugu, çift numarali sayfa
\textwidth 16cm  %metin genisligi
\textheight 20cm %metin yüksekligi
\linespread{1} %Satir araliklarini ayarlar. özel olarak ;{\setlength{\baselineskip}
                                                        %{1.5\baselineskip}
                                                        %Bu paragraf baseline skip çarpanini
                                                       %1.5 alarak dizilmistir. Paragraf
                                                       %sonundaki komuta dikkat edin.\par}




%
% Begin
%
\begin{document}
%
% Kapak
%
\thispagestyle{empty}
\setcounter{page}{0}
\begin{center}
\textbf{\Huge{Bilim İnsanının Amerika'da PhD'ye Başvuru Rehberi} \\
\vspace{8mm}
\Large{The Hitchhiker's Guide to Graduate School \\
\vspace{4mm}
Applications in the US }}\\

\vspace{74mm}
\large{Onur Albayrak} 
\\
ve
\\
Tuna Toksöz 
\\
ve 
\\
Alp Sipahigil 
\\
ve 
\\
Uğur Güney 
\\
ve
\\
Buğra Kaytanlı \\
\vspace{6mm}
\small{
Editör 
\\
Tolga Suna }
\end{center}
%
% Lisans
%
\section*{Lisans}

Bu çalışma Creative Commons Attribution-NonCommercial-Share-Alike 3.0 ile lisanslanmıstır. Bu çalışmayı kopyalamak, dağıtmak veya herhangi bir basılı veya elektronik ortamda göstermek, atıfta bulunulduğu sürece serbesttir. Bu çalışma ticari amaçla kullanılamaz. Yapılacak olan değişiklikler aynı lisans altında yapılmalıdır. 

Lisansın tam metnini aşağıdaki linkten bulabilirsiniz

http://creativecommons.org/licenses/by-nc-sa/3.0/legalcode

\byncsa
\newpage
%
% Ithaf
%
\vspace*{45mm} 
\textit{Bu rehber saygıdeğer hocamız Prof. Alpar Sevgen'e ithaf edilmiştir.}
%
\newpage
%
% icindekiler
%
\tableofcontents
%
\newpage

%
% ONSOZ
%
\section{Önsöz}
Sevgili gençler, 

Bu rehberi yazma düşüncem başvuru sürecinde çekilen dertler sonrasında oluştu. Umarım bu rehber hepinizin işine yarar ve güzel yerlere kabul alırsınız. Daha eklenecek birçok şey olabilir. Aklıma geldikçe ve/veya kabulünü almış, eleğini asmış gençler rehbere eklemeler yaparsa ne güzel olur. Metin düzenlemelerini yapan kadim dostum Tolga Suna'ya da teşekkür ediyorum. Hepinizi sevgi ve saygı ile selamlıyor, şimdiden başarılar ve iyi şanslar diliyorum. 

\noindent OA \\
Ağustos, 2010 \\

\noindent Sevgili gençler, 

Bu rehberi yazma düşüncem vardı fakat Onur benden önce davranıp başlamış bile. Özellikle başvuru aşamasında, başvuru sonrasında okulun istedikleri belgeler ve Türkiye'den ayrılmadan yapılması gereken işler konusunda size yardımcı olacak bu kaynağı hazırlamak için kolları sıvadık. Bu doküman her ne kadar Amerika eksenli olsa da, içeriğin büyük bir kısmının Kanada, Avrupa, Avustralya, Uzak Doğu ve Mars'taki üniversiteler için de geçerli olduğunu düşünüyoruz. Bu yol çetin fakat bir o kadar da eğlenceli bir yol. Hepinize bol şanslar.   

\noindent TT \\
Ağustos, 2010 \\

\textit{Not: Çeşitli kısımlar, çeşitli kişiler tarafından yazıldığından birinci tekil konuşmalar sizleri şaşırtabilir. Bu rehberde yazılanlar tamamen yazarların düşünce\-leridir. İşverenlerin görüşlerini yansıtmaz. Rehberdeki içeriğin kullanımından doğabilecek zararlardan yazarlar sorumlu değildir. Bu rehberdekiler bir tavsiye niteliğinde alınmalıdır.
Rehberin en güncel haline \href{https://github.com/yurtdisiphdrehberi/yurtdisiphdrehberi/}{bu adresten} ulaşabilirsiniz.}

\newpage
%
%
%
\section*{Yazarlar ve okulları}

Onur Albayrak, \textit{Carnegie Mellon University - Physics }

Tuna Toksöz, \textit{Massachusetts Institute of Technology - Aeronautics and Astronautics }

Alp Sipahigil, \textit{Harvard University - Physics }

Uğur Güney, \textit{City University of New York - Physics }

Buğra Kaytanlı, \textit{University of California, Santa Barbara - Mechanical Engineering}
\newpage
%
% Yaz stajlari, Bugra
%
\section{2-3. Sınıflarda Araştırma ve Yaz Stajları}
Bu kitapçığın ileri bölümlerinde arkadaşlarım, başvuru sürecinde çok önemli olduğuna inandığmız konulara değiniyorlar. Ben de bunlara ek olarak dördün\-cü sınıfa gelmeden önce yapılabilecek hazırlıklardan bahsedeceğim.

Phd’ye kabul almak için gerekli en temel üç yatırım unsuru vardır: GPA, “recommendation letter”lar ve “prior research experience” denilen şey yani stajlar, “workshop”lar vesaire. Bu kitapçığı okuyorsanız zaten derslerinize geçen iki-üç yıl boyunca gayet sıkı çalışmış notlarınızı yüksek tutumuşsunuzdur ki, şu sıralarda  PhD yapma fikri aklınızdan geçiyordur. O yüzden GPA kısmını atlayıp, geri kalan iki kısımla ilgili birkaç noktaya değineceğim.

\subsection{Tavsiye Mektubunun İçinin Doldurulması}
Hocalardan tavsiye mektubu istemek, hocaların seçimi, “nasıl söylesem” dertleri her birimizin en az birkaç gece uykularını kaçırmıştır. Nitekim uykularımız boşa kaçmamıştır çünkü tavsiye mektubu mühim bir iştir. Mektupları isteyeceğiniz hocalarınızı sizi (mümkünse kişiliğinizi de) en iyi tanıdığına inandığınız hocalarınızdan seçmeniz faydalı olacaktır. Mesela dersine hiç aksatmadan gittiğiniz ve sonunda sınıftaki iki AA’dan birini aldığınız bir dersin hocasından tavsiye mektubu istemek çok yerinde olacaktır. -mı acaba?- Bir hoca tavsiye mektubuna ne yazar ki? Başvuruları okuyan komite bu mektupları neden ister ve bu mektuplarda ne görmek ister? Sanırım bu sorunun cevabı “çok zor bir dersten birçok parlak öğrencinin arasında AA aldığınızı” vurgulayan bir metin yerine “sizin ders materyalinin dışında hocaya neler sorduğunuzu, kendisiyle ne tarz çalışmalar içine girdiğinizi, diğer AA alanlardan ne farkınız olduğunu” vurgulayan bir metin olacaktır. Burada şunu anlamamız gerekir ki tavsiye mektuplarını yazanlar hocalarımızdır. Fakat bu mektupların içini dolduranlar biz öğrencileriz. Bu yüzden ikinci üçüncü sınıflarda iken konusuna uzaktan bir aşinalığınız olan, içinizde küçük de olsa bir merak uyandıran bir hocanıza gidip bu konularla ilgilenmek istediğinizi söyleyebilirsiniz. Zamanla severseniz (büyük ihtimalle seveceksinizdir) devam edersiniz, sevmezseniz başka bir hocanıza sorabilirsiniz. Hocalarınız bundan alınmayacaklar size karşı bir tavır almayacaklardır, çünkü neredeyse hiçbir insanoğlu hayatının ileri evrelerinde 19-20 yaşlarında ilgi duyduğunu sandığı işi yapmıyor. (en azından benim için öyle oldu) Yani işin özü şudur ki, dördüncü sınıf başlayıp da “recommendation letter” isteme zamanı geldiğinde kendimizi ve mektubu istediğimiz hocamızı “dersime hiç aksatmadan geldi, en başarılı oldu ve AA aldı, çok gayretli bir çocuk” demek zorunda bırakmak hocamızı ve kendimizi ve sıkıntılı bir duruma sokacaktır; öyle ki çok başarılı bir öğrenci olduğu halde AA aldığı bir dersin hocasına gittiğinde hocamızın inanılmaz dürüst ve olgun tavrı sayesinde “ben sana etkili bir mektup yazamam ki, başka bir hocandan istesen çok daha iyi edersin” cevabını alan bir arkadaşım da olmuştur. Kısacası; bu işi son ana bırakmayın, hocalarınızın araştırmalarını küçük sınıflarda iken araştırmaya başlayın.

\subsection{Yaz Stajları Hakkında}
Bulunduğunuz okulda birçok farklı grupta research deneyimi kazanmış ya da eğitim hayatınız boyunca hiç böyle bir imkan bulamamış bile olabilirsiniz. Her iki durumda da (tercihen yurtdışında) başka bir okulda bir araştırmaya katılmanız çok faydalı olacaktır. Öyle ki, birçok “admission committe” üyesinin söylediği-yazdığı şey şu ki: “a 4.0 with no prior research experience wouldn’t get you into PhD”. Bunun için ise öncelikle aklınıza ilk gelen okullardaki ilgilendiğiniz departmanların sitesine girip hocalara ve araştırma konularına bakmakla başlayabilirsiniz. Hoca araştırmasında Türkleri seç\-meye çalışabilirsiniz, ben öyle yaptım. Çoğu zaman yurtdışındaki Türk hocalar laboratuarlarında parlak ve istekli Türk öğrencilere bir görev bulmakta daha yardımsever oluyorlar, tabi ki burada yabancı hocalara ulaşmaya çalışmayın demiyorum. Böyle yaz stajı yaptığını bildiğim beş kişiden ikisi Türk hocanın yanında diğerleri yabancı hocaların lab’larında çalışmıştı. 

\subsubsection{Diğer Departmanlardan Korkmayın}
Diyelim staj yapmayı çok istediğiniz bir okul/enstitü var ama bu okulda sizin major’ınızda\-ki hocaların hiçbirinin araştırması size çekici gelmiyor, veya Türk bir hoca aradınız ve bulamadınız, hemen yelkenleri suya indirmeyin ve lütfen diğer departmanlara da bir göz atın. Şimdi ben fizikçiyim elektrikçinin işinden ne anlarım, makineciyim biyomedikalde ne yapabilirim ki, kimyacıyım CS departmanıyla ne işim olur demeyin\footnote{Sadece belki matematikçi değilseniz matematik departmanına bakmaktan biraz çekinebilirsiniz.}.  Tuna Boğaziçi’nde bilgisayar okurken MIT’de aero-astro bölümünde staj yaptı, ben fizik-makine okurken MIT’de Research Lab of Electronics’de nöron kestim biyoloji araştırması yaptım, başka bir arkadaşım endüstri okurken Yale’da “environmental sciences” departmaninda staj yaptı.

\subsection{Staj Ayarlamak}
Ben üçüncü sınıftan dörde geçerken stajımı 3.sınıfın Şubat ayında halletmiştim. Baharın gelmesini beklemeden, ilk dönem midtermlerinize girip onlarla işinizi bitirdikten sonra sömestr tatilini hoca aramak, bulmak, mail yollamak, cevap beklemek, sonunda da stajınızı ayarlamakla geçirebilirsiniz. Ayrıca şunu da unutmamak lazım ki, eğer siz bir hocaya ulaşabilmiş ve ona mail yollamışsanız bu hoca büyük ihtimalle bir websitesi, research grubu olan dersler veren, konferanslara giden, “grant application”ları yazan çok meşgul bir insan olacaktır. Ve aynı insana sizin gibi en aşağı 157 tane Hintli, Çinli, Koreli, İngiliz hatta Türk da mail atıyor. O yüzden mail optimum uzunlukta olmalı, yani sizin nasıl bir background’unuz olduğu ve adamın konusuna ne kadar ilgi gösterdiğinizi açıklayabilecek kadar uzun fakat aynı zamanda adamın sıkılıp okumaya baymayacağı kadar da kısa olmalı. Bence gmail penceresinde yaklaşık 6 satırlık bir mail işinizi görecektir.

\subsubsection{Pes Etmemeli}
Attığınız maillerin 50'de birine cevap bekleyin, o da çok büyük ihitimalle "sorry we don't have an opening for the summer, good luck in your search" ya da "I will not take anymore students  this summer" olacaktır. Sakın pes etmeyin! Bir arkadaşım için beş yüze yakın hocaya mail attık ve sadece bir kişiden olumlu cevap geldi: "We might find a spot for you in the lab". Bunu yazan hoca da Yale'dan bir (Amerikalı) hocaydı. Sonuç olarak yaz geçip başvurular tamamlanıp kabul zamanı geldiğinde arkadaşım birçok pek de bilinmeyen okuldan ret aldı ama Yale'da o hocanın yanına “environmental sciences” master programına girdi.

\subsection{Başvuru Stratejisi}
Diyelim ki okuduğunuz bölümle direk olarak alakalı olan bir işle değil de daha çok başka bir bölümün alanına giren bir konuya ilgi duyuyorsunuz. Mesela ben biyomedikalle ilgilenmek istiyordum ve bu yüzden on tane okula biyomedikal mühendisliği bölümüne başvurdum on tane ret aldım ve sadece bir okula kendi bölümüm olan makine mühendisliğine başvurdum ve oradan da kabul aldım. Ve danışmanım fizik PhD’li olup biyomedikal araştırması yapıyor, ben de şu anda makine mühendisliği öğrencisi olmama rağmen nere\-deyse hiçbir makine dersi almayıp onun yerine istediğim departmanlardan derslerimi alarak biyoloji alanında çalışıyorum. Yani burada benim kendi deneyimimden yola çıkarak tavsiye edebileceğim şey şudur ki başvuru safha\-sına geldiğinizde bölüm değiştirmek yerine kendi bölümünüzde hoca olan fakat ilgilendiğiniz konuda çalışan bir hoca bulmanız avantajınıza olabilir, çünkü diğer departmanda yanına başvurduğunuz hocanın yaptığı işe çok uygun olsanız ve sizi almak istese bile alamayabilir. Çünkü son kararı bir komite veriyor ve komite "bu delikanlının altyapısı bizim departmana pek de uygun değil, qualifierlar’da çok sancılı zamanlar yaşayacaktır" diyebilir.

\subsubsection{Lisans Sürecinde Araştırma Yapmak}
Sakın ben daha ikinci sınıfım ne bileceğim, daha diferansiyeli yeni aldım diye korkmayın, diferansiyel ve temel fizik dersleri birçok problemi anlamada yeterli oluyor bence. Zaten size bir lisans öğrencisi olarak verilecek is çok ağır olmayacaktır, kendimde ve etrafımda gördüğüm manzaralar hep bu yönde oldu. Mesela ben ne yaptım: kimyasalları birbirine karıştırdım, doktora öğrencilerinin özenle hazırlamış oldukları karmaşık set-up’larda operatör gibi buna bas şuna bas yaptım. Peki, bu yaz stajı bana tam olarak ne kattı? Araştırma nasıl yapılır, ne zorlukları vardır, araştırmada sorunla karşılaşmak nedir, üstesinden nasıl gelinir (ya da gelinemez) gibi konularda enikonu fikir edindim. Birçok makale okudum ve bilimsel makale nasıl aranır elde edilir onu öğrendim. Her şeyden en önemlisi de gerçekten biofizik alanını istiyor muyum yoksa benim için geçici bir heves miymiş onu anladım (istiyormuşum). Çünkü çoğu zaman bilim alanında insanların yaptığı secimler altı çok da dolu olmayan kulaktan dolma seçimler oluyor. Bu yüzden ilgi duyduğunuzu düşündüğünüz alanda ciddi bir akademik staj sadece resume'nizde yakışıklı duracağı için değil aynı zamanda sizin bu işi gerçekten sevip sevmediğinizi anlamanız için de çok önemli bir adım olacaktır.

\subsection{Yaz Stajının Finansmanı - The Dark Side of The Moon}
Yaz stajından para kazanmayı beklememek gerekiyor. Tüm giderlerinizin sizin ve ailenizin üzerinde olacağı varsayımıyla yola çıkmak yerinde olacaktır. Özellikle de çok top bir okulda bir hocanın yanına gidecekseniz ne yardan ne serden dememek gerekiyor bu noktada. Amerika’da 3 ay için 3*800 (kalma masrafı) + 1500 (yemek masrafı) = 3500-4000 dolar civarı para ayırmanız gerekebilir. Ama tabi ki size para ödeyecek staj hayatta bulamazsınız demiyorum. Bulabilirseniz çok iyi olur ama bu işlerde default olan 2.5-3 ay para almadan çalışmak. Çünkü şöyle düşünülebilir;  adamın okulunda, elinin altında zaten istekli onlarca öğrenci var, adam neden laboratuarına international bir öğrenci getirtmeye uğraşsın ki? Ayrıca, PhD’ye girdiğinizde 5 yıl boyunca bütün masraflarınız karşılanacak ve aylık 2000 dolar mertebesinde, kiranızı ödemenize, yemeğinizi yemenize, üstünüze başınıza birkaç öteberi almanıza ve az da olsa sağı-solu gezmenize yetecek bir maaşınız olacak.

\subsubsection{Amerika’daki Pahalı Staj’a Muadil Olarak}
Türkiye’de okullar var Koç'ta çalışan hocalar var onlara mail atabilirsiniz. Boğaziçi makinede okuyup Koç makinedeki bir hocayla araştırma yaptığını bildiğim bir arkadaşım var (şimdi UIUC’de). Sabanci ve Bilkent'te de çok iyi arastirma yapan hocalar var onlara da bakabilirsiniz. Avrupa’da Eindhoven, ETH gibi okullarda araştırma imkanları olduğunu biliyorum, oralarda staj yapan insanlar da oldu etrafımda.

\subsection{Bölümdeki Hocalara Yaz Stajını Danışmak}
Ayrıca bölümdeki hocalarınıza da staj ile ilgili yardımcı olup olamayacakları sorulabilir: “bir tanıdığınız-beni tavsiye edebileceğiniz, bağlantı kurabileceğiniz hatta beni yollayabileceğiniz bir yer var mı” diye. Benim bir hocam (ben stajımı ayarladıktan çok sonra): "keşke bana soyleseydin, seni şuraya yollardık orda bizim Hüseyin var hem para da verirdi" dedi, ama benim için iş işten geçmişti tabi. Yani şunu fark etmeliyiz ki hocalarımız “biz talep ettiğimiz sürece” biz talebelere beklediğimizden çok daha fazla imkanlar yaratabilirler.

\subsection{Son Olarak}
Yaz  stajıyla ilgili son olarak da şöyle bir gözlemde bulundum:  Deneysel çalışan laboratuvarlarda çok fazla staj imkanı oluyor çünkü bu laboratuarlar “hands-on” iş gücüne ihtiyaç duyarlar ve bu tarz gruplarda pozisyon bulmanız nispeten daha kolay olabilir. 
\newpage
%
% Basvuru taktikleri
%
\section{Başvuru Zamanı İçin Tavsiyeler}
%
\subsection{Kendine Göre Okul ve Hoca Aramak}
Okulları bulduktan sonra başvurunda bahsedeceğin ya da mail atacağın hocayı seçmek önemli bir iştir. Bu iki iş uzunca vaktinizi aldığı gibi başına oturup odaklanması da zordur. İyisi mi siz internette sörf eylerken bir yandan bu işlere bakıyor olun. Bu araştırmaların, Ekim-Kasım zamanı artık sonlanması lazım ki, tam başvurular esnasında yeni okullar ortaya çıkmasın. Başvuru işini ranking üzerinden okul seçip yapıyorsanız; “bugün 10-15 arasına bakacağım” gibi planlar yapılabilinir. 

Okulları bulmak bence işin en önemli kısmı. Bu işin sonunda kesinlikle, üniversite web sitelerinin uzmanı oluyorsunuz. Onur'un dediği gibi boş vakit mi buldunuz, internette poker oynayacağınıza (lafımın kime olduğu açık), research gruplarının sitelerinde dolanın. Dolandıkça insanın kafasında ``yahu bu adamın yaptığı şeyler iyiymiş'' gibi fikirler yavaş yavaş oturuyor.. 

Diyelim A adlı arkadaşı (arkadaş dediğim hoca) sevdiniz. Son publish ettiği yayınlara bakın, beraber çalıştığı insanlara bakın. Standart bir yöntem şu olabilir: Elimizde süperstar bir fizikçi var (attım Randall, bizim köylü de ondan). Kasış bir okulda olduğu için kendisi çok meşgul. Bu yüzden ne yapıyoruz? Kadının en son parlamakta olan doktora öğrencisi ya da beraber çalıştığı kişi iyicene bir üniversitede canavar gibi yeni fikirler üretmektedir. Biz de hemen Randall'la ortak yayın yaptığından dolayı zınk diye yakalıyor, “aha ben bu okula da girerim” deyip o okulu araştırıyor, potansiyel hocayı gözümüze kestiriyor ve listemize o okulu ekliyoruz. 

Boğaziçi'nin nerelerle nasıl bağları olduğunu anlamak da çok yararlı. Yakın olduğunuz hocalarla konuşun bu konuyu, onların tanıdıklarının olduğu yerlerde insan birkaç adım önde başlıyor. Birkaç adım demek çok adım demek. 

Kaliforniya Sendromu'na yakalanmayın. Fizik'ten bildiğim kadarıyla University of California sisteminden kimseye kabul gelmedi. Bir tek esaslı oğlan Bugra Kaytanli'yi aldılar. İlla Kaliforniyacı iseniz o zaman USC, Caltech, Stanford gibi Kaliforniya'da olan özel okullara başvurun. Tabi çok isteyenler yine Santa Barbara için şansını deneyebilir. 

Şehir faktörü: Burada Boston'u öveceğim biraz. Kısaca akademik bir cennet Boston, günün her saati her konuda bir konferans yapılıyor. 

Enstitü faktörü: Bu da mühim bir konu. İlgilendiğiniz alanla ilgili bir enstitünün üniversite dahilinde bulunması ortamın ne kadar canlı olduğunu gösteriyor. Örneğin, Maryland'de JQI, UCSB'de Kavli, Berkeley'de Lawrence, Tufts'ta Cosmology... Enstitüleri bu okulların belirli alanlarda önder olmasını sağlıyor. 

İşin devamı ile ilgili tavsiyelere geçmeden önce bu okul seçme kısmına bol bol değinmek lazım. Olaya sadece ranking olarak yanaşmak bence verimsiz. Kendimden birkaç örnek daha vereyim.  
Ben mesela Cornell ve Princeton fizik departmanlarına başvururken içimden ``ya aslında ben bunların yaptığı işlere pek ısınmadım'' diyordum. Ama bunu paylaşmaktan haliyle çekiniyordum, “ulen sen kim oluyorsun da ısınmıyorsun” derler diye. Bizim ilgi alanımızla departmanın eğilimlerinin uyuşması çok önemli. Onlar da benim için "bu oğlan fena değil ama bize gitme'' dediklerini tahmin ediyorum. Bence iyi de ediyorlar çünkü bu karar sizin için de iyi oluyor falan falan. 

Kısacası CTRL+D kombinasyonuna alışın, bookmarklarınız grup siteleri ile dolsun.  

\subsection{Sınavlar, Sınavların Alım İşleri}
%ekleme1
Temelde okulların çoğu 2 sınav istiyor. Biri İngilizce yeterliliğinizi ölçtüğünü sanan TOEFL, diğeri ise ne işe yaradığı hala bilim çevrelerince tartışılan GRE. Sınavlar için tarihleri almak, yok efendim gün seçimi, mekan seçimi derken bir kaç gün alabiliyor. İki sınavınız arasında 10 gün bırakmanız iyi olacaktır ki çok sıkışıp sizi strese sokmasın. Ancak temel bilimlere başvuracaksanız GRE Subject sınavına da girmeniz gerekebilir. 


\subsubsection{TOEFL}
TOEFL 4 bölümden oluşuyor: Reading, Listening, Writing ve Speaking. Hangi bölüm kaç sorudan oluşuyor hatırlamıyorum, fakat aralarda bonus sorular çıkabiliyor. Yani Listening'den 30 soru çıkması gerekirken 45 soru çıkabilir. Bunların hangilerinin puanlanacağını sınav sırasında bilemiyoruz, fakat sınav çıkısında sizinle giren arkadaşlara sorarsanız kesişimi bulabilirsiniz. Reading klasik reading, sorular sırasında solda paragraf açık duruyor, yani parçayı önceden bir okuyayım notlar alayım yapmanıza gerek yok. Listening de normal listening. Bir şeyler dinliyorsunuz ve sonra sorulan sorulara cevap veriyorsunuz. Burada not almak önemli. Speaking'de 6 tane speaking sorusu var. Bunların 2 tanesi havadan sudan şeyler, “hayatında gördüğün en güzel kızı tanımla'', “en çok neden kor\-karsın'' gibi bir şeyler. Kalanları ``integrated'' dediğimiz Listening ile birleştirip yapılacak sorular. Önemli olan yine not alabilmek. Speaking'te benim gördüğüm zaman doldurma için tekrar yapabilirsiniz. Her ne kadar 45 saniye kısa gibi görünse de bazen söyleyecek bir şey gelmiyor. Bu durumda, “mm” diyip daha önce söylediğiniz bir cümleyi ``paraphrase'' edip zaman kazanabilirsiniz.  Fakat bu tekrarları çok yapmayın, kafa ütülemeyin. Writing ise iki tane, biri klasik writing, konu veriliyor ve bununla ilgili orta uzunlukta bir essay yazmanız isteniyor. Diğeri ise yine integrated. TOEFL için çalışılabilecek 3-4 kaynak var. Kaplan, Cambridge, Barron veya Longman. Yanlış hatırlamıyorsam Longman en kolayı. Daha sonra Kaplan, sonra Barron ve sonra da Cambridge. Ancak Barron's gerçek sınava en yakın olan diyebilirim. 

\subsubsection{GRE General}
GRE, General çok boş bir sınav. 3 bölümden oluşuyor: Quantitative, Verbal, Writing. Quantitative bizim lise seviyesinde (3/2+1/5)*x=7 ise “x’in yarısı kaçtır?'' şeklinde sorular. ÖSS çabukluğunu kaybettiğinizi düşünüyorsanız 1-2 test çözün hem nasıl sorular olduğunu anlarsınız. En kıl tüy Q soruları grafikli olanlar, yok grafiği nerede kesiyor, orası 75 civarı mı yoksa 77.5 un biraz üstü gibi mi sorular. GRE Verbal inanılmaz ezber isteyen bir bölüm. 3600 tane kelimesi mi ne var, ezberle ezberle bitmez. O yüzden pek uğraşmayın. Writing önemli. 2 tane writing var. Birinde konu veriliyor siz yazıyorsunuz, diğerinde ise bir adamın söylediği önermenin ve arkasındaki ``reasoning''in neden yanlış olduğunu yazdığınız bir essay. GRE general çalışmaya değecek bir şey değil, en azından temel bilimci olan gençlerimiz için, mühendislik isteyenlerin biraz daha iyi olsa iyi olabilir belki.  

\subsubsection{GRE Subject}
GRE Subject sınavı senede 3 kere yapılıyor; Ekim, Kasım ve Nisan aylarında. "Ekim ayındaki sınavın sonucunu öğreneyim ona göre Kasım ayında sınavı tekrar alırım" deme şansınız olmuyor çünkü kayıtlar sınavdan neredeyse bir ay önce kapanıyor. 


\subsubsection{Sınavların açıklanması ve sonuç belgeleri}
Burada dikkat edilmesi gereken bir husus, GRE General ve GRE Subject sınavlarına kayıt olurken tamamen aynı isim, soyad, doğum tarihi ve bilgilerinizi girmenizin gerekliliği. Yoksa GRE'nin sistemi aynı kişi olduğunuzu bir türlü anlamıyor ve Subject ve General'ı beraber gönder komutunu veremiyorsunuz. Bu da gönderi masraflarının ikiye katlanması demek. GRE Subject sınavı senede 3 kere yapılıyor; Ekim, Kasım ve Nisan aylarında. "Ekim ayındaki sınavın sonucunu öğreneyim ona göre Kasım ayında sınavı tekrar alırım" deme şansınız olmuyor çünkü kayıtlar sınavdan neredeyse bir ay önce kapanıyor. 


GRE ve TOEFL sistemleri için şifrenizi bir yere not edin. Saçma gelebilir ama koydukları şifre politikaları yüzünden şifrenizi unutursanız ``Forget Password'' yaparken hesabınız sebepsiz yere kilitlenebilir. Daha sonra telefonla saatlerinizi geçirirsiniz açtırmak için. 

TOEFL sınav sonucu 14 gün içerisinde internette açıklanıyor. Sonuç kağıdının evinize varması ise daha uzun sürebilir. GRE General sınavını da 14 gün içerisinde internetten öğrenebilirsiniz. Ancak bu sürede sadece Quantitative ve Verbal kısımlarının sonuçları açıklanıyor. Yaklaşık 21 gün sonra Writing kısmı da açıklanmış oluyor. GRE Subject skorlarının açıklanması ise 6 haftayı buluyor. Ancak 4 hafta içerisinde telefon ile arayıp skoru erken öğrenme şansınız mevcut. Bu hizmet için ETS 12\$ alıyor. Okulların başvurularını yaparken sonuç belgelerinin size veya okullara varmış olmasına gerek yok. İnternetten sonucu görebiliyor olmanız, başvuruyu doldururken gerekli kısımlara sonuçlarınızı yazmanız yeterli. ETS sonuçları toplu şekilde, yanılmıyorsam ayda iki kere, okullara dağıtıyor. 

Sınav sonuçları geldikten sonra ``additional score report''ların gönderimi, bilgisayar başında geçen bir iki saat diyebilirim. Bazen daha da uzun sürebiliyor; kredi kartınız sorun çıkartıyorsa ve çok okula başvurduysanız. Hepsini seçmek, hepsini onaylamak derken bu gerçekten cins bir iş. Ama bunu elbette ki Kasım ayı civarında yapmanız lazım. 

GRE skor belgesi yollarken hem General hem Subject seçip gönderebiliyorsunuz. Ayrı ayrı sipariş vermeye gerek yok. Ayrıca birden fazla girdiğiniz sınavların bir tanesini yollamak gibi bir şey olmuyor. Ben iki tane subject sınavına girmiştim, skor belgelerinde her iki sınav ve genel sınav sonucu yazıyordu. 

GRE'yi aramanız gerekirse Skype üzerinden ücretsiz hatlarını 888 ile başlayan, ya da paralı hattı arayabilirsiniz. Amerika aramaları için Skype kredisi çok ucuz oluyor.  

\subsection{Okulların Online Başvurularının Doldurması}
Bu iş kesinlikle en cins işlerden biri, her okulun kendine has soruları olduğu için en uzun zaman alan işlerden birisi budur. Bu işe girişmeden önce okulları belirlerseniz daha güzel, daha hoş! Bir Word dosyası gibi bir şey yaratın ve genel bilgilerin hepsini buna yazın: Adresiniz, okulunuzun adı, referansların isim, adres ve mail bilgileri, sınav sonuçlarınız, sınav kodlarınız vs. Copy paste kullanarak bu işi çok hızlandırabilirsiniz. Bu işe istediğiniz an başlayabilirsiniz çünkü neredeyse tüm okullar doldurulan formları kaydedip sonradan kalınan yerden devam etme imkanı veriyor. Kullanıcı adı ve şifrelerinizi çok alakasız seçmeyin 10-15 okulla uğraşırken hepsine farklı isim kullanırsanız ``of deli miyim ben'' diyeceksiniz.. Google Chrome bazı okullarda sorun çıkarttığından tüm okulların başvurularını Firefox'ta, ``şifreyi hatırla'' seçeneğini seçerek yapmanız tavsiye edilir.  

Bazı cins okullar ``one shot'' mantalitesinde, yani bir oturumda her şeyi doldurmanız gerekli. Bu tür okullar admission sayfalarında bunu söylüyorlar zaten. Bunları en sona bırakın. Ne de olsa tüm sonuçların gelmiş olması SOP (statement of purpose)'un yazılmış olması gerekli. 

\subsection{SOP Yazılması}
Bu işe toplamda bir ay verebilirsiniz, çünkü ha yaptım, ha yapacağım derken zaman geçiyor. SOP'unuzda ``küçüklüğümden beri çöpçü olmak istiyorum, alın beni''; şeklinde bir şey yazmayın, yemezler. Mümkün olduğu kadar akademik geçmişinizden bahsedin, laboratuvar çalışmaları, projeleriniz ve projelerinizin başvurduğunuz departmanla ilişkisinden bahsedin. Ayrıca temeli oluşturduktan sonra her başvurduğunuz okula göre değiştireceğiniz bir kısım da olacak. Onu sona bırakabilirsiniz. Bu okula özel kısmın gerçekten okula özel olmasına önem gösterin. MIT'yi sileyim, yerine Stanford yazayım derseniz eli\-nizde kalabilir, zira okulu yazarken lab da yazarsınız fakat Stanford'da öyle bir lab yoktur falan filan. Şu projeler ilgimi çekti yazacaksanız da o projenin 5 yıl önce tamamlanmış olmadığından emin olun. SOP yazdığım süreçte ben versiyon numarası kullanmıştım, küçük değişiklikler oldu mu sop1.0’dan sop1.1’e geçiyordum. Büyük değişikliklerde sop1.1’den sop2.0’a geçiyordum ve tüm dosyaları da sakladım. Böylece geriye dönüşler daha kolay oluyor. Bölümünüzde daha once bu tip işlerle uğraşan biri varsa SOP'unuzu ona veya hocalarınıza da okutabilirsiniz. Onların öğretmen gözünden bakması faydalı olacaktır. Bazı yerlerde hoca ismi yazmanın faydalı olacağını okumuştum ama bilmiyorum, bu size kalmış. (Onur'un notu: Hoca ismi belirtmek iyi olacaktır diye düşünüyorum. Hem bölümü araştırdığınızı göstermek açısından, hem de bir hedefe kitlenmişlik imajı vermesi açısından önemli. Ancak yine de çok kesin laflar etmemek gerekiyor.) Ayrıca yaptığınız okul dışı etkinliklerden küçük bir paragraf içinde bahsedebilirsiniz. Örneğin, ``efendim ben yelken yaptım ve takım ruhunu yakaladım; ki bu da araştırma grubun da çalışırken büyük bir avantajdır'' şeklinde  kendinizi pazarlayın. 

\subsection{Başvuru Dosyalarını Hazırlamak, Göndermek}
Başvuru sürecinde en önemli noktalardan biri de dosyaların yedeklenmesi ve el altında bulunması. dropbox.com adresindeki online storage sistemini öneriyorum. Bilgisayarınıza bir program yüklüyorsunuz ve bir klasör seçiyorsunuz. Bu klasörün içindeki dosyalar online olarak yedekleniyor ve hem web üzerinden hem de başka bilgisayarlarda bu program yüklü ise kullanıcı adı ve şifrenizi girerek klasörün tüm içeriğine ulaşabilirsiniz. Hem yedekleme işlevi ile başvuru sürecinde bilgisayarınıza bir şey olursa içiniz rahat olur, hem de her yerden erişebilirsiniz. 

Diğer bir konu ise emailinizi en etkili şekilde kullanmak. Mesela TOEFL sonucunuz geldi, tarattınız ve bilgisayarınızda var. Hemen kendinize ya da bir başka emailinize konu başlığı ``TOEFL SCORE REPORT PDF'' olan bir mail atın. Bu sayede internet erişiminiz olan her yerde email hizmetinizin arama kısmını kullanarak kolayca sonuç pdf'nizi bulup bir yerlere upload edebilir ya da email ile iletebilirsiniz. Keyword kullanımı çok önemli. Mesela alakasız bir yerdeyim, bir okuldan ``TOEFL gelmemiş gönderebilir misiniz?'' dediler. Hemen aramamı yaptım bunu forwardladım iki saniyede: ``Beyin bedava''. 

Ayrıca hocalara mail atarken okul mail adresinizi kullanmaızı tavsiye ediyorum. Eğer kullandığınız mail hizmetinde kullanıcı adınız isim ve soyadınızdan çok farklı ise (Örnek: crazyboy85@gmail, iwannadoPhD@gmail) maillerinizin spam'i boylaması muhtemel olacaktır. Okul mailimi kullanarak attığım mailler, gmail'ime göre cevap süresi farkedilebilir oranlarda azaldı. (4-5 günden, 1-2 güne). "Okul mail sistemimiz çok kötü kullanasım gelmiyor.'' diyorsanız, maili gmail'inize yönlendirebilir oradan mail alışverişinde bulunabilirsiniz. Diğer bir tavsiyem ise "*@*.edu'' adreslerinden gelen maillerin spam klasörünüze düşmesini engellemeniz. Bunu mail sağlayıcınızın filtre ayarlarını, mailin geldiği kısma "*@*.edu'' yazıp bu mailleri spam klasörüne düşmeyecek şekilde düzenlemek. ("*''lar her türlü edu ile biten mailleri anlatmak için kullanılmıştır.)

Başvuracağınız okullardaki hocalara mail atıp fikir sorma işini ise Kasım-Aralık aylarına doğru yapmanız daha iyi olacaktır. Çünkü adamlara mailinizde bahsedebileceğiniz sınav sonuçlarınız vs. gibi bilgiler de elinize geçmiş olur. Fakat şunu da unutmayın, muhtemelen 10 milyon Çinli de o hocaya mail atıyor olacaktır. 

Ayrıca başvurularda transkript isteniyor. Eğer 5 taneden fazla için para yatırırsanız iki katı kadar veriliyor. 10 parası yatırıp 20 tane aldım ve onlarla başvuruları yaptım.  

Başvuruların belgelerini yollama işlerine gelince, birkaç gün önceden gidip FEDEX ya da DHL'den adres formlarını alın evde doldurup mekana o şekilde gidin. Böylece yollama işlerini hızlandırabilirsiniz. FEDEX'in daha iyi çalıştığını ve daha ucuz olduğunu gördük, bize öyle denk gelmiş de olabilir. FEDEX’in merkezi güney girişindeki otobüs durağının orada bulunan HİSAR COPY. Bu arada belirtilmesinde fayda olan diğer bir nokta ne kadar çok gönderimi aynı anda yaparsanız toplam gönderim masrafınızın o kadar düşeceği. Toplu gönderimlerde indirim yapıyorlar. 
%
%
%
\newpage
\subsection{Başvuru Masrafları}
Okullara başvurmak ve sonrası oldukça masraflı bir iş.  

Başvurduğunuz okul sayısından bağımsız masraflar: 
\begin{center}
\begin{tabular*}{0.5\textwidth}{@{\extracolsep{\fill}}  l r}
TOEFL & 185\$ \\

GRE General & 190\$ \\

GRE Subject & 160\$ \\

Vize Randevu & 20\$ \\

Vize Başvuru &140\$ \\

SEVIS & 200\$ \\ 
\end{tabular*} \\
\end{center}

Başurduğunuz okul sayisi ile doğru orantılı masraflar: 
\begin{center}
\begin{tabular*}{0.5\textwidth}{@{\extracolsep{\fill}}  l r}
Okul başvuru ücreti (Ortalama) & 75\$ \\ 
TOEFL Additional Score Report & 17\$ \\ 
GRE Additional Score Report & 23\$ \\ 
Transkript gönderme masrafı & 19 Euro \\
\end{tabular*} \\
\end{center}


Burada masrafları biraz olsun azaltmak için dikkat edilmesi gereken birkaç nokta var. Örneğin TOEFL ve GRE'ye başvururken veya sınavdan hemen sonra 4 okula skor gönderme işini ücretsiz yapabiliyorsunuz. Ayrıca GRE + Subject sonuçlarını beraber gönderebilirsiniz. 15 okula başvurduğunuzu varsayarsak (vize ücretleri hariç) yaklaşık ortalama $180+190+160+75*15+17*11+23*11+19*1.4*15=2494\$ $. Vize ile ilgili ücretleri de hesaba kattığımızda bu rakam 2860\$ gibi rakamlara ulaşabiliyor. 
\newpage
%
% Kabul sonrasi
%
\section{Kabul Sonrası}
\subsection{Askerlik İşleri}

Askerlik şubesine mümkün olduğunca erken gitmeye çalışın. İçeriye cep telefonu alınmıyor ancak telefonu kapalı şekilde oraya emanet edebiliyorsunuz. 8.30'da iş başı yapılıyor ve o saatte kimsecikler olmuyor. Ancak fotokopici daha geç açıldığından ışıkların yan tarafında bulunan Adliye'de fotokopi çektirebilirsiniz. İşlemler için 4 adet vesikalik, 4 diploma (arkalı önlü), ve 4 adet arkalı önlü nüfus cüzdanı fotokopisi gerekiyor. Vesikalığınızın size benziyor olmasına ve çokeski olmamasına dikkat edin. Ayrıca Nüfus Cüzdanınız eskiyse fotonuz küçüklükten kalmaysa, şubede işleminizi yapanların moduna göre reddedilebilirsiniz. Bunun riskine girmemek için nüfus cüzdanınızı yeniletebilirsiniz. Muhtarlıktan alacağınız bir kağıt aldıktan sonra sonra nüfus müdürlüğüne gitmeniz gerek. Ufak bir cüzdan masrafı alıyorlar o kadar. Muhtarlıktan kağıt almadan giderseniz yanınızda ikinci bir kimlik olmasını isteyebilirler. Bunun için pasaport veya ehliyet kullanabilirsiniz. Nüfus müdürlükleri Kaymakamlıklar'da bulunuyor. Beşiktaş ilçesindeki, Çırağan Sarayı\-'nın karşısında Emniyet Müdürlüğüne gelmeden önce. 

Sıra numaranızı aldıktan sonra görevli sizi bina içindeki şubeye yönlendiriyor. Orada işlemlerinizi tamamladiktan sonra (imzalar, ailenle sorunun var mı, komando olmak ister misin gibi çeşitli sorular vs.) sizi hastaneye sevk edecekler. Zarfınizı alıp hastanenin yolunu tutuyorsunuz. Yine sahil yolunu takiben Kasımpaşa Askeri hastanesi bulunmakta. Hastanede kapıda sizi yönlendiriyorlar.Binaya girdikten sonra, danışmaya size verilen zarf ile başvuruyorsunuz. Orada ilk imzayı aldıktan sonra görevli sizi diğer hekime yönlendiriyor. Sizi yönlendirdikleri birim, ameliyat vs. nedeniyle o saatte açık değilse ya da bir başka nedenle çok dolu ise, görevliye ``başka bir birim imzalayabilir mi'' derseniz daha boş bir yere yönlendiriyorlar. Doktor size şöyle bir bakıp ``tamam iyisin'' diyor ve işlem bitiyor. Fazla kilo ya da başka sorunlarınız varsa belirli bir birime yönlendirebilirler. (Arkadaşı önce geri şubeye yeni belgeler almaya oradan da iç hastalıklara yönlendirdiler. Bu yüzden kilo veya başka bir probleminiz varsa yanınızda mutlaka fazladan fotoğrafınız bulunsun. Mümkünse 8-10 tane.) Sonrasında baştabibin sırasına girip görevli askere evraklarınizı veriyorsunuz, o toplu sekilde imzalatip getiriyor. Sonra tekrar askerlik şubesine geri dönüyorsunuz. 

Askerlik şubesinde sıra numarası vs. almadan direkt sizinle ilgilenen memura gidip veriyorsunuz belgelerinizi, birkaç imza sonrasında size verilen belgeyi alıp, mutlu şekilde evinize dönebilirsiniz. 

\subsection{Pasaport İşleri}
Yeni pasaport almak için, öncelikle internetten e-pasaport.gov.tr adresini kullanarak size en yakın Emniyet Müdürlüğü'nden randevu almanız gerekiyor. Bizim zamanımızda randevu almayanları da sıraya alabiliyorlardı ancak bu işlemler tamamen online sisteme geçeceği ve tam saatinde sizi içeri alma avantajını kullanmak için online sistemi kullanın. Yeni defter ücretinizi, emniyete gitmeden önce Ziraat Bankası'na yatırmanız gerekiyor. Diğer bankalara yatırılan paralar kabul edilmiyor. Yeni çekilmiş 5*5 boyutlarında vesikalık götürmeniz gerekiyor. Avrupa için vize alınan gibi değil! Amerikan vizesi için gereken ile aynı. Başvuruya giderken varsa eski geçerli pasaportunuzu da götürüp onun iptalini de orada gerçekleştiriyorsunuz. Nüfus cüzdanınızı da unutmayın. Online randevu aldığınız internet sayfasında neler gerektiğinin listesini bulabilirsiniz. Ayrıca ne kadar senelik alacağınıza göre değişen pasaport harç makbuzunuzu da yanınızda götürmeniz gerekli. Doktora için gidecekler 5 senelik ücretsiz alma şansını kullanabilir. Bunun nasıl olacağı bir başka paragrafta açıklanıyor, biraz sabredin. Yeni düzenleme ile 10 senelik pasaport da alabiliyorsunuz ancak bunu ücretsiz alma şansınız ne yazık ki yok. 10 senelik almak isteyebilirsiniz çünkü yeni çipli pasaportların kullanma süresi dolduktan sonra yeniden uzatılamıyor ve yeni bir defter veriliyor. (Ağustos 2010 için geçerli bir düzenleme). Ayrıca Üniversite Kabul Mektubu’nuzu yanınızda götürmeyi unutmayın Pasaport başvurumu Beşiktaş İlçe Emniyet Müdürlüğü’nde gerçekleştirdim. Sıra ve hizmet Şişli’ye göre daha iyi. 

Çipli pasaport başvurunuz sonrasında 7-10 işgünü içerisinde evinize postalanıyor ve imzanız ile almanız gerekiyor. 

Öğrenci olduğunuz ve eğitim/kültür amacıyla gittiğiniz için harç masrafı ödemeniz gerekmiyor. Pasaport harcından muaf olabilmek için yapılması gerekenlerden bahsedeceğim. Vergi Dairesi tecrübeleri kişiden kişiye çok değişiyor çeşitli örneklerden bahsedeceğim. Ben Boğaziçi'den aldığım ``Bu öğrenci doktoraya gidecektir, harctan muaf olsun'' şeklindeki yazı, kabul mektubum ve nüfus cüzdanım ile gitmiştim. Adamlar Boğaziçi'nden aldığım yazının geçerli olmadığını çünkü artık o üniversite ile ilişkim kalmadığını, kabul mektubunun yeminli tercümesini getirmem gerektiğini belirttiler. Tercüme ile gelirsen 5 yıl ücretsiz veririz dediler. Anlaşıldığı üzere Vergi Dairesi\-nde işler günden güne ve orada sizin ile ilgilenen adama göre değişiyor. Ayrıca daha önce birçok arkadaş Boğaziçi'nden aldıkları ``Bu öğrenci doktoraya gidecektir harçtan muaf olsun'' yazısı ile gittiler Vergi Dairesi'ne ve işlerini hallettiler. Yasaya göre kabul mektubunun çevirisi gerekiyor ve en fazla 2 yıl ücretsiz pasaport hakkı verilebiliyor. Ama daha önce de dediğim gibi kişiden kişiye yasanın uygulaması değişiyor.  Halihazırda pasaportunuz varsa bunu 2015 yılına kadar kullanabilirsiniz. Aynı şey Emniyet Müdürlüğü'ne Vergi Dairesi'nden götürdüğünüz kağıtta da geçerli. Emniyet kabul mektubunuzu isteyecek. Orijinalinin taranmış halinin print-out'u yeterli oluyor. Tercümeyi Şişli Adliyesi'nin etrafındaki tercüme bürolarında yaptırtabilirsiniz. Tüm mektubu çevirtmenize gerek yok, kabul edildiğiniz kısmı ve kaç sene süreciğini yazan kisimlar yeterli olacaktır. Ve tabii mektubu imzalayanın, kısım üniversitenin isminin kısımlarını da atlamayın. İki paragraf için yaklaşık 15 TL tutuyor. (Haziran 2010) 

\subsection{Vize İşleri}
Amerika vize konusunda her şeyi sisteme oturtmuş, işler çok rahat yürüyor. Sitesinde yeterli açıklama, gerekli belgeler mevcut. Yapmanız gereken I-20 Formunuz geldikten ve pasaportunuzu aldıktan sonra en kısa zamanda DS160 formunu internetten doldurmak. Eskiden 3 ayrı form vardı, şimdi hepsini tek forma indirdiler. Oldukça uzun ve sıkıcı bir form. Kimlik bilgileri, daha önce bulunduğunuz ülkeler, sizi tanıyan insanlara ait bilgiler gibi çeşitli bölümleri mevcut. Bu formu doldururken en kilit nokta her sayfayı doldurduktan sonra ``save'' etmeniz. Zira arada webserver application pool'u restart olduğu için sizin bilgileriniz uçabilir. Ayrıca bu dosyayı saklarsanız sonraki vize basvurularınızda kullanabilirsiniz. (Örneğin; master’dan sonra doktora yaparken tekrar başvurmanız gerekiyor.) Vize formunu doldurduktan sonra SEVIS ücretini yatırmanız gerekiyor. SEVIS ücreti internette https://www.fmjfee.com/i901fee/ adresinden kart ile ödeniyor. Posta ile gelen dekonta gerek olmadan, ödedikten hemen sonraki çıktıyı vize mülakatı sırasında konsolosluğa götürebilirsiniz. Formu doldurup SEVIS'i yatırdıktan sonra yapmanız gereken konsoloslugu 0212-340-4444 telefonundan arayıp randevu almak. O zamanki yoğunluğa göre randevuyu 1 ay sonraya verebilirler. Şanslıysanız 3-4 gün sonraya da verebiliyorlar. Bu yüzden bu işleri önceden halletmekte fayda var. Konsolosluğa gitmek için Boğaziçi'nden taksiye binebilirsiniz. Yaklaşık 13 TL tutmuştu sanıyorum ki. Randevu saatiniz 9:15 ise siz 8:45'te orada olun çünkü çoğu zaman önceden alıyorlar. Konsolosluğa giderken yanınızda götürmeniz gereken birkaç şey var. Bunlardan biri DS160 formunu doldurduktan sonra karşınıza çıkan barkod sayfası. Eğer DS160 formuna resminizi ekleyemediyseniz yanınızda bir iki adet vesikalık da götürmeniz gerekiyor. Ayrıca okuldan gelen I20 formu, pasaport, SEVIS dekontu, transkriptleriniz de gerekli. Eğer okulunuzdan burs almıyorsanız okulun ilk senesine yetecek kadar paranızın olduğunu gösteren Banka Müşteri Temsilcisi tarafından imzalanan, bankanın antetli kağıdı üzerine İngilizce yazılmış Bank Statement. Bunun dışında tapu falan da götürebilirsiniz ama bakmıyorlar. Eğer bursunuz varsa fakat bursunuz 9 aylıksa sizden geri kalan 3 aylık kısmı da göstermenizi isteyebilirler. Bu da yaklaşık 6-7 bin dolar göstermeniz demek. Genelde I20, SEVIS, DS160 ve pasaport dışında hiçbir şeye bakmamaktadırlar. 
%
%
%
\end{document}